%HHHHHHHHHHHHHHHHHHHHHHHHHHHHHHHHHHHHHHHHHHHHHHHHHHHHHHHHHHHHHHHHHHHHHHHHHH

\section{Understanding the "right-hand side"\label{RHS}}

%HHHHHHHHHHHHHHHHHHHHHHHHHHHHHHHHHHHHHHHHHHHHHHHHHHHHHHHHHHHHHHHHHHHHHHHHHH

As can be seen from the above example, you need to know the exact appearance
of the "right-hand side" 
in order to correctly write your semantic action.
Your method must be prepared
to handle a varying configuration that depends on the parsed text.
To have this configuiration in front of you while writing the code,
it is convenient to describe it by a comment as shown above.
(The author has learned this style from his colleague Bertil Steinholtz,
who used it to write very clear \tx{yacc++} code.) 

The objects appearing on the "right-hand side" 
correspond to the names and terminals appearing to the right of \tx{"="}
in the rule.
You can find the possible configurations of the "right-hand side"
like this:

\smallskip
\fbox{\quad\parbox{0.935\linewidth}{\upsp 
Take the expression to the right of \tx{"="}
and remove from it all predicates, that is, sub-expressions 
of the form $\text{\&}e$ and $\text{!}e$.
Replace each \tx{"/"} by \tx{"|"}, \tx{"*+"} by \tx{"*"},
and \tx{"++"} by \tx{"+"}.
The result is a regular expression on an alphabet consisting
of names and terminals treated as single letters.
The possible configurations of the "right-hand side"
are exactly the strings defined by this regular expression.\dnsp}\quad }

\medskip
As an example, the regular expression obtained for \Sum\ is:

\small
\begin{Verbatim}[samepage=true,xleftmargin=15mm,baselinestretch=0.8]
 Number ("+" Number)*
\end{Verbatim}
\normalsize

which indeed defines these strings of symbols \Number\ and \tx{"+"}:

\small
\begin{Verbatim}[samepage=true,xleftmargin=15mm,baselinestretch=0.8]
 Number "+" Number ... "+" Number
   0     1    2        n-2  n-1
\end{Verbatim}
\normalsize

where $n = 2p + 1$ for $p = 0, 1, 2, \ldots\;$.

