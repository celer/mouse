\newpage   
%HHHHHHHHHHHHHHHHHHHHHHHHHHHHHHHHHHHHHHHHHHHHHHHHHHHHHHHHHHHHHHHHHHHHHHHHHH

\section{Appendix: Your parser class\label{DocPars}}

%HHHHHHHHHHHHHHHHHHHHHHHHHHHHHHHHHHHHHHHHHHHHHHHHHHHHHHHHHHHHHHHHHHHHHHHHHH

These are the methods you can apply to your generated parser.\newline
"\tx{Parser}" and
"\tx{Semantics}" are names of your parser and semantics classes,
respectively.

\ul

\item[\textbf{Parser}\texttt{()}]\upsp \newline
   Parser constructor. Instantiates your parser and semantics,
   connects semantics object to the parser,
   and returns the resulting parser object.\dnsp

\item[\texttt{boolean }\textbf{parse}\texttt{(Source src)}]\upsp \newline
   Parses input wrapped into a \tx{Source} object \textit{src}.\newline
   Returns \tx{true} if the parse was successful, or \tx{false} otherwise.\dnsp

\item[\texttt{Semantics }\textbf{semantics}\texttt{()}]\upsp \newline
   Returns the semantics object associated with the parser.\dnsp

\item[\texttt{void }\textbf{setTrace}\texttt{(String s)}]\upsp \newline
   Assigns $s$ to the \tx{trace} field in semantics object.\dnsp

\item[\texttt{void }\textbf{setMemo}\texttt{(int n)}]\upsp \newline
   Sets the amount of memoization to $n$, $0 \le n \le 9$.\newline
   Can only be applied to a parser generated with option \tx{-M} or \tx{-T}
   (see \tx{mouse.Generate} tool).

\eul
