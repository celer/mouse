\newpage
%HHHHHHHHHHHHHHHHHHHHHHHHHHHHHHHHHHHHHHHHHHHHHHHHHHHHHHHHHHHHHHHHHHHHHHHHHH

\section{Input from file\label{file}}

%HHHHHHHHHHHHHHHHHHHHHHHHHHHHHHHHHHHHHHHHHHHHHHHHHHHHHHHHHHHHHHHHHHHHHHHHHH

In more serious applications, like a compiler, the parser has to read 
input from a file.
We are going to show the way of doing this by an adaptation
of the calculator developed above.

Just suppose you want to perform a sequence of calculations such as this:

\small
\begin{Verbatim}[samepage=true,xleftmargin=15mm,baselinestretch=0.8]
 n = 0  
 f = 1
 e = 1
 e
 n = n + 1 
 f = f * n
 e = e + 1/f  
 e
 n = n + 1 
 f = f * n
 e = e + 1/f  
 e
   ...
\end{Verbatim}
\normalsize

that will print the consecutive stages of computing the number $e$
from its series expansion.
You want the calculator to read this input from a file.
All you need is to enlarge your syntax by defining the input
as a sequence of \tx{Line}s:

\small
\begin{Verbatim}[frame=single,framesep=2mm,samepage=true,xleftmargin=15mm,xrightmargin=15mm,baselinestretch=0.8]
   Input = [\n\r]* Line+ EOF ;
   Line  = Space (Store / Print) ;
   Store = Name Space Equ Sum EOL {store} ;
   Print = Sum EOL {print} ;

    ...  as before  ...   

   EOL     = [\n\r]+ / EOF <end of line> ;
   EOF     = !_ <end of file> ;
\end{Verbatim}
\normalsize

This grammar defines \tx{EOL} to be either end of line
or end of input.
End of line is defined as one or more newlines and/or carriage returns,
meaning that the syntax allows empty lines -- really empty, without
even space characters in them.
(The \verb#"\n"# and \verb#"\r"# stand for newline and carriage
return characters, respectively; see Section~\ref{escape}.)
The \verb#[\n\r]*# at the beginning permits the file to start with empty lines.

You do not need to change semantics.

Copy \tx{myGrammar.txt} and sample file \tx{e.txt} from \tx{example9},
generate the parser and compile it in the usual way.

To run the parser on the sample file, type at command prompt:

\small
\begin{Verbatim}[samepage=true,xleftmargin=15mm,baselinestretch=0.8]
 java mouse.TryParser -P myParser -f e.txt
\end{Verbatim}
\normalsize

The option \tx{-f e.txt} instructs the tool to take its input from
the file \tx{e.txt}.
You should get this output:

\small
\begin{Verbatim}[samepage=true,xleftmargin=15mm,baselinestretch=0.8]
 e.txt
 1.0
 2.0
 2.5
 2.6666666666666665
 2.708333333333333
 2.7166666666666663
 2.7180555555555554
 2.7182539682539684
 2.71827876984127
 2.7182815255731922
 2.7182818011463845
 --- ok.
\end{Verbatim}
\normalsize

The \Mouse\ tool \tx{TryParser} is provided to help you in developing the parser,
but eventually you will need to invoke the parser from one of your own programs.
The following is an example of how you can do it:

\medskip
\small
\begin{Verbatim}[frame=single,framesep=2mm,samepage=true,xleftmargin=6mm,xrightmargin=6mm,baselinestretch=0.8]
 import mouse.runtime.SourceFile;
 class FileCalc
 {
   public static void main(String argv[])
     {
       myParser parser = new myParser();          // Instantiate Parser+Semantics
       SourceFile src = new SourceFile(argv[0]);  // Wrap the file
       if (!src.created()) return;                // Return if no file
       boolean ok = parser.parse(src);            // Apply parser to it
       System.out.println(ok? "ok":"failed");     // Tell if succeeded
     }
 }
\end{Verbatim}
\normalsize

The class \tx{SourceFile} supplied by \Mouse\ is a wrapper
that maps the file in memory 
and contains methods needed to access it and to
print error position in diagnostics.
You find the above code as \tx{FileCalc.java} in \tx{example9}.
Copy it into the \tx{work} directory and compile.
You can then invoke the calculation by typing:

\small
\begin{Verbatim}[samepage=true,xleftmargin=15mm,baselinestretch=0.8]
 java FileCalc e.txt
\end{Verbatim}
\normalsize

To see the kind of diagnostics produced for a file, damage \tx{e.txt}
by, for example, duplicating one of the operators 
or changing \tx{"n"} to \tx{"x"} in an expression,
and run the parser on the damaged file.
