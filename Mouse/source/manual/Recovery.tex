%HHHHHHHHHHHHHHHHHHHHHHHHHHHHHHHHHHHHHHHHHHHHHHHHHHHHHHHHHHHHHHHHHHHHHHHHHH

\section{Error recovery\label{errRec}}

%HHHHHHHHHHHHHHHHHHHHHHHHHHHHHHHHHHHHHHHHHHHHHHHHHHHHHHHHHHHHHHHHHHHHHHHHHH

If you tried the parser on a faulty file, you have noticed that it stops
after encountering the first faulty line.
It perhaps does not make sense for this particular computation,
but in many cases you would like the parser to just skip the faulty line
and continue with the next.
For example, if you use
the parser to implement a compiler, you want it
to continue and possibly discover more errors.
Such behavior can be achieved by defining a faulty line as an additional
alternative for \tx{Line}, to be tried after \tx{Store} and \tx{Print} failed.
With this addition, the first lines of your grammar would look like this
(note the use of the shorthand operator \tx{++}):

\small
\begin{Verbatim}[frame=single,framesep=2mm,samepage=true,xleftmargin=15mm,xrightmargin=15mm,baselinestretch=0.8]
   Input   = [\n\r]* Line++ EOF ;
   Line    = Space (Store / Print) ~{badLine}
           / BadLine ;
   Store   = Name Space Equ Sum EOL {store} ;    // store result
   Print   = Sum EOL {print} ;                   // print result
   BadLine = _++ EOL ;                           // skip bad line
\end{Verbatim}
\normalsize

The \tx{BadLine} is defined to  be any sequence of characters
up to the nearest end of line or end of input.
After a failure of \tx{(Store / Print)}, the parser backtracks
and consumes all of \tx{BadLine}.

The name in brackets preceded by a tilde: \verb#~{badLine}# identifies 
a semantic action to be invoked in case of failure.
Thus, \tx{badLine} is invoked when
the first alternative of \tx{Line} fails.
The method has an access to a Phrase object for \tx{Line},
appearing as the left-hand side of the failing rule
\tx{Line = (Store / Print)}.
There is no right-hand side, but \tx{lhs()} contains
error information.
The function of \tx{badLine} is to print out this
information and erase it.
This is done by calling helper methods 
\tx{errMsg()} and \tx{errClear()}:

\small
\begin{Verbatim}[frame=single,framesep=2mm,samepage=true,xleftmargin=15mm,xrightmargin=15mm,baselinestretch=0.8]
   //-------------------------------------------------------------------
   //  Line = Store / Print (failed)
   //-------------------------------------------------------------------
   void badLine()
     {
       System.out.println(lhs().errMsg());
       lhs().errClear();
    }
\end{Verbatim}
\normalsize

(The line will not be rescanned, so it is safe to print the message.)

The use of \tx{++} in \tx{Input} prevents a repeated call to \tx{BadLine}
after an erroneous last line.
Because \tx{BadLine} is called in an iteration, 
it must consume at least one character.
It is therefore defined using \tx{++} rather than \tx{*+}.
You may verify that it will never be called at \tx{EOL}.

You can find the modified grammar and semantics in \tx{example10},
together with the source for \tx{FileCalc}.
Generate the new parser in the usual way and compile everything;
then damage the input file in several places,
and run the parser on it using \tx{FileCalc}
to see the diagnostics.
