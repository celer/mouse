%HHHHHHHHHHHHHHHHHHHHHHHHHHHHHHHHHHHHHHHHHHHHHHHHHHHHHHHHHHHHHHHHHHHHHHHHHH

\section{Get error handling right\label{errors}}

%HHHHHHHHHHHHHHHHHHHHHHHHHHHHHHHHHHHHHHHHHHHHHHHHHHHHHHHHHHHHHHHHHHHHHHHHHH

Try the following with the calculator just constructed:

\small
\begin{Verbatim}[samepage=true,xleftmargin=15mm,baselinestretch=0.8]
 java Calc
 > 2a
 After '2': 'a' is not defined
 NaN
 > a = 12(
 After 'a = 12': expected Space or Sum
 > a
 12.0
 > 2a)
 24.0
 After '2a': expected [a-z] or Space or MultOp or Factor or AddOp or end of text
 > 
\end{Verbatim}
\normalsize

Note that the input \tx{"a = 12("} resulted in failure,
but response to the subsequent input \tx{"a"} shows that \tx{"a"}
has been defined.
The input \tx{"2a)"} resulted again in failure,
but the result of \tx{"2a"} has been printed.

What happens?

In the first case, \Store\ successfully consumed \tx{"a = 12"},
and its action \Storea\ entered \tx{"a"} into the dictionary with \tx{12} as value.
Then it returned success to \Input\, which failed by finding \tx{"("} instead of end of text,
backtracked, and issued the error message.
%
In the second case, \Print\ successfully consumed \tx{"2a"}, printed its result,
and returned succes to \Input\, wich failed by not finding end of text.

The moral of the story is that \textbf{you must not take any irreversible actions
in a situation that can be reversed by backtracking}.

A simple solution in this case is by making the predicate \tx{"!\_"} part 
of \Store\ and \Print:

\small
\begin{Verbatim}[frame=single,framesep=2mm,samepage=true,xleftmargin=15mm,xrightmargin=15mm,baselinestretch=0.8]
   Input   = Space (Store / Print) ;
   Store   = Name Space Equ Sum !_ {store} ;
   Print   = Sum !_ {print} ;
\end{Verbatim}
\normalsize

In this way, \Storea\ and \Printa\ are called only after the entire input
has been successfully parsed.

\medskip
You may wonder why the message after \tx{"a = 12("} says nothing about the expected end of text.

A non-backtracking parser stops after failing to find an expected character
in the input text, and this failure
is reported as \emph{the} syntax error.
A backtracking parser may instead backtrack
and fail several times.  
It terminates and reports failure when no more
alternatives are left.
The strategy used by \Mouse\ is to report only the failure that occurred 
farthest down in the input.
If several different attempts failed at the same point,
all such failures are reported.

In the example, the message says that \tx{Lparen} expected blanks after \tx{"("},
and that the third alternative of \Factor\ expected \Sum\ at that place.
The failure to find end of text occurred earlier in the input and is not mentioned.

\medskip
In the case of \tx{"2a)"}, several procedures failed immediately after \tx{"2a"}.
The processing did not advance beyond that point, so all these failures are reported.

You may feel that information about 
\tx{Space} not finding a blank is uninteresting noise. 
To suppress it, you can define the following semantic action for \tx{Space}:
%
\small
\begin{Verbatim}[frame=single,framesep=2mm,samepage=true,xleftmargin=15mm,xrightmargin=15mm,baselinestretch=0.8]
   //-------------------------------------------------------------------
   //  Space = " "*
   //-------------------------------------------------------------------
   void space()
     { lhs().errClear(); }
\end{Verbatim}
\normalsize
%
When \tx{space()} is invoked, the \tx{lhs()} object seen by it
contains the information that \tx{Space()} ended
by not finding another blank.
The helper method \tx{errClear()} erases all failure information
from the \tx{lhs()} object, so \tx{Space} will never report any failure.

If you wish, you can make messages more readable by saying
\tx{"+ or -"} instead of \tx{"AddOp"}.
You specify such "diagnostic names" in pointed brackets at the end of a rule, like this:

\small
\begin{Verbatim}[frame=single,framesep=2mm,samepage=true,xleftmargin=15mm,xrightmargin=15mm,baselinestretch=0.8]
   AddOp   = [-+] Space  <+ or -> ;
   MultOp  = [*/]? Space <* or /> ;
   Lparen  = "(" Space   <(> ;
   Rparen  = ")" Space   <)> ;
   Equ     = "=" Space   <=> ;
\end{Verbatim}
\normalsize

A grammar and semantics with these modifications are found in \tx{example8}.
Copy them to the \tx{work} directory, generate new parser (replacing the old), and compile both.
A parser session may now appear like this:

\small
\begin{Verbatim}[samepage=true,xleftmargin=15mm,baselinestretch=0.8]
 java Calc
 > 2a
 After '2': 'a' is not defined
 NaN
 > a = 12(
 At start: 'a' is not defined 
 After 'a = 12': expected Sum
 > a
 At start: 'a' is not defined 
 NaN
 > 2a)
 After '2': 'a' is not defined
 After '2a': expected [a-z] or * or / or Factor or + or - or end of text
 >
\end{Verbatim}
\normalsize

The message about \tx{"a"} not being defined in response to \tx{"a~=~12("} 
is another example of an irreversible action
in a situation reversed by backtracking.
It appears because \Print\ is attempted after the failure of \Store,
and issues the message before itself failing.
Perhaps a better strategy should be invented for reporting undefined names,
but it does not seem disturbing, and we leave it as it is.
Note, however, that it would be entirely wrong if we reversed the order of \Store\ and \Print.
In that case, a session would look like this:

\small
\begin{Verbatim}[samepage=true,xleftmargin=15mm,baselinestretch=0.8]
 java Calc
 > a = 12
 At start: 'a' is not defined
 > a
 12.0
 > 
\end{Verbatim}
\normalsize

Here, the parser applies \Print\ first, reports \tx{"a"} undefined, fails by finding \tx{"="},
backtracks, and successfully executes \Store.


