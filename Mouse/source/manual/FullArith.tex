%HHHHHHHHHHHHHHHHHHHHHHHHHHHHHHHHHHHHHHHHHHHHHHHHHHHHHHHHHHHHHHHHHHHHHHHHHH

\section{Full arithmetic\label{arith}}

%HHHHHHHHHHHHHHHHHHHHHHHHHHHHHHHHHHHHHHHHHHHHHHHHHHHHHHHHHHHHHHHHHHHHHHHHHH

As the next development, we suggest adding multiplication, division
and parentheses:

\small
\begin{Verbatim}[frame=single,framesep=2mm,samepage=true,xleftmargin=15mm,xrightmargin=15mm,baselinestretch=0.8]
   Input   = Space Sum !_ {result} ;
   Sum     = Sign Product (AddOp Product)* {sum} ;
   Product = Factor (MultOp Factor)* {product} ;
   Factor  = Digits? "." Digits Space {fraction}
           / Digits Space {integer}
           / Lparen Sum Rparen {unwrap} ;
   Sign    = ("-" Space)? ;
   AddOp   = [-+] Space ;
   MultOp  = [*/]? Space ;
   Lparen  = "(" Space ;
   Rparen  = ")" Space ;
   Digits  = [0-9]+ ;
   Space   = " "* ;
\end{Verbatim}
\normalsize

We have replaced here \Number\ by \tx{Factor} that can be 
a fraction, an integer, or a sum in parentheses.

We have added \tx{Product} with multiplication and division
to ensure, in a syntactical way, the priority of these operations
over addition and subtraction.
The omitted operator in \tx{MultOp} is intended to mean multiplication.

As \Sum\ can now appear in \tx{Factor},
we had to add \Input\ as new top level, 
moving there the initial \Space\ and the check for end of input.

The semantic action \Suma\ cannot now print its result;
instead, the result will be inserted as semantic value 
into the \Sum's result,
to be retrieved by the semantic action of \Input\ or \Factor:

\small
\begin{Verbatim}[frame=single,framesep=2mm,samepage=true,xleftmargin=15mm,xrightmargin=15mm,baselinestretch=0.8]
   //-------------------------------------------------------------------
   //  Sum = Sign Product AddOp Product ... AddOp Product
   //          0     1      2      3         n-2    n-1
   //-------------------------------------------------------------------
   void sum()
     {
       int n = rhsSize();
       double s = (Double)rhs(1).get();
       if (!rhs(0).isEmpty()) s = -s;
       for (int i=3;i<n;i+=2)
       {
         if (rhs(i-1).charAt(0)=='+')
           s += (Double)rhs(i).get();
         else
           s -= (Double)rhs(i).get();
       }
       lhs().put(new Double(s));
     }
\end{Verbatim}
\normalsize

This assumes, of course, that \Product\ delivers its result as semantic value
of its result.
The semantic action \Producta\ follows the same pattern as \Suma:

\small
\begin{Verbatim}[frame=single,framesep=2mm,samepage=true,xleftmargin=15mm,xrightmargin=15mm,baselinestretch=0.8]
   //-------------------------------------------------------------------
   //  Product = Factor MultOp Factor ... MultOp Factor 
   //               0     1      2         n-2     n-1
   //-------------------------------------------------------------------
   void product()
     {        
       int n = rhsSize();
       double s = (Double)rhs(0).get();
       for (int i=2;i<n;i+=2)
       {
         if (!rhs(i-1).isEmpty() && rhs(i-1).charAt(0)=='/')
           s /= (Double)rhs(i).get();
         else  
           s *= (Double)rhs(i).get();
       }  
       lhs().put(new Double(s));
     }
\end{Verbatim}
\normalsize

(\tx{MultOp} 
means division if it is a nonempty string starting with with \tx{"/"};
otherwise it is multiplication.)

Note that the precedence of multiplication and division 
over addition and subtraction is here defined by the syntax.
However, the order of operations within each class (left to right)
is \emph{not} defined by the syntax, but by semantic actions.

The actions \tx{fraction()} and \tx{integer()} 
remain the same as in the preceding example.
The function of \tx{unwrap()} is to take the semantic value of \Sum\
and deliver it as the semantic value of \Factor:

\small
\begin{Verbatim}[frame=single,framesep=2mm,samepage=true,xleftmargin=15mm,xrightmargin=15mm,baselinestretch=0.8]
   //-------------------------------------------------------------------
   //  Factor = Lparen Sum Rparen 
   //              0    1    2
   //-------------------------------------------------------------------
   void unwrap()
     { lhs().put(rhs(1).get()); }
\end{Verbatim}
\normalsize

No casts are needed.
It remains to print the result:

\small
\begin{Verbatim}[frame=single,framesep=2mm,samepage=true,xleftmargin=15mm,xrightmargin=15mm,baselinestretch=0.8]
   //-------------------------------------------------------------------
   //  Input = Space Sum !_
   //            0    1
   //-------------------------------------------------------------------
   void result()
     { System.out.println((Double)rhs(1).get()); }
\end{Verbatim}
\normalsize

The modified grammar and semantics are available in \tx{example5};
you may copy them 
into the \tx{work} directory,
then generate and compile the parser as before.
The session may now look like this:

\small
\begin{Verbatim}[samepage=true,xleftmargin=15mm,baselinestretch=0.8]
 java mouse.TryParser -P myParser
 > 2(1/10)
 0.2
 > (1+2)*(3-4)
 -3.0
 > 1.23(-4+.56)
 -4.2312
 >
\end{Verbatim}
\normalsize

Notice that the multiplication operator can, in particular, be the empty string.
This would cause an ambiguity in the classical context-free grammar,
where \tx{"234"} could be parsed as meaning \tx{2*3*4}, \tx{23*4}, or \tx{2*34}.
However, there is no place for ambiguity in the parser specified by PEG:
it will always read the string as integer \tx{234}.
Which is, by the way, exactly how a human reader sees it.

