%HHHHHHHHHHHHHHHHHHHHHHHHHHHHHHHHHHHHHHHHHHHHHHHHHHHHHHHHHHHHHHHHHHHHHHHHHH

\section{Introduction: recursive-descent parsers with backtracking}

%HHHHHHHHHHHHHHHHHHHHHHHHHHHHHHHHHHHHHHHHHHHHHHHHHHHHHHHHHHHHHHHHHHHHHHHHHH

Parsing Expression Grammar (PEG),
introduced by Ford in \cite{Ford:2002,Ford:Thesis,Ford:2004},
is a new way to specify a recursive-descent parser with limited backtracking.
 
Recursive-descent parsers have been around for a while.
Already in 1961, Lucas \cite{Lucas:1961} suggested the use of recursive procedures
that reflect the syntax of the language being parsed.
This close connection between the code and the grammar
is the great advantage of recursive-descent parsers.
It makes the parser easy to maintain and modify.

A recursive-descent parser is simple to construct
from a classical context-free grammar
if the grammar has the so-called $LL(1)$ property;
it means that the parser can always decide what to do
by looking at the next input character.
However, forcing the language into the $LL(1)$ mold 
can make the grammar -- and the parser -- unreadable.

The $LL(1)$ restriction can be circumvented by the use of backtracking.
Backtracking means that the parser proceeds by trial and error:
goes back and tries another alternative if it took a wrong decision.
However, an exhaustive search
of all alternatives may require an exponential time.
A reasonable compromise is limited backtracking, 
also called "fast-back" in \cite{Hopgood:1969}.

Limited backtracking was adopted in at least two of the early top-down designs:
the Atlas Compiler Compiler 
of Brooker and Morris \cite{Brooker:Morris:1961,Rosen:1964},  
and TMG (the TransMoGrifier) of McClure \cite{McClure:1965}.
The syntax specification used in TMG was later formalized and analyzed
by Birman and Ullman \cite{Birman:1970,Birman:Ullman:1973}.
It appears in~\cite{Aho:Ullman:1972} as "Top-Down Parsing Language" (TDPL)
and "Generalized TDPL" (GTDPL).
Parsing Expression Grammar is a development of this latter.

The name "Grammar" may be confusing when applied to what 
is essentially a top-down parsing language,
but it was introduced in \cite{Ford:2004} as a new,
recognition-based, method of defining syntax.
It also has an appearance similar to grammars in the
Extended Backus-Naur Form (EBNF) --
although this similarity can often be misleading.

Parsers defined by PEG
do not require a separate "lexer" or "scanner".
Together with the lifting of the $LL(1)$ restriction,
this gives a very convenient tool when we need 
an ad-hoc parser for some application.

Even the limited backtracking may require a lot of time.
In \cite{Ford:2002,Ford:Thesis}, PEG was introduced together with
a technique called \emph{packrat parsing}.
Packrat parsing handles backtracking
by extensive \emph{memoization}: storing all results
of parsing procedures\footnote{
"Packrat" comes from \emph{pack rat} -- a small rodent (\emph{Neotoma cinerea})  
known for hoarding unnecessary items; also a person that does the same.
"Memoization", introduced in \cite{Michie:1968}, is the technique  
of reusing stored results of function calls instead of recomputing them.}.
It guarantees linear parsing time at a large memory cost.
There exists a complete parser generator named \textsl{Rats!} \cite{Grimm:2004,Grimm:Rats}
that produces packrat parsers from PEG.

Excessive backtracking does not matter 
in small interactive applications
where the input is short and performance not critical.
Moreover, the usual programming languages
do not require much backtracking.
%
Experiments reported in \cite{Redz:2007:FI,Redz:2008:FI}
demonstrated a moderate backtracking activity  
in PEG parsers for programming languages Java 1.5 and C.

\medskip
\Mouse\ is a development of parser generator used for these experiments.
It translates PEG into a set of recursive procedures that closely follow the grammar.
Unlike \textsl{Rats!}, \Mouse\ does not produce a packrat parser.
Optionally, it can offer a small amount of memoization using the technique
described in \cite{Redz:2007:FI}.
Both \Mouse\ and the resulting parser are written in Java.
%
An integral feature of \Mouse\ is the mechanism for specifying
semantics (also in Java). 

\medskip
After a short presentation of PEG in the following section,
the rest of the paper has the form of a tutorial,
introducing the reader to \Mouse\ by hands-on experience.
