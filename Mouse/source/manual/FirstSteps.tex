\newpage
%HHHHHHHHHHHHHHHHHHHHHHHHHHHHHHHHHHHHHHHHHHHHHHHHHHHHHHHHHHHHHHHHHHHHHHHHHH

\section{The first steps}

%HHHHHHHHHHHHHHHHHHHHHHHHHHHHHHHHHHHHHHHHHHHHHHHHHHHHHHHHHHHHHHHHHHHHHHHHHH

We start with the sample grammar from Section~\ref{PEG}.
You find it as the file \tx{myGrammar.txt} in \tx{example1}.
Copy it to the \tx{work} directory,
and enter at the command prompt:

\small
\begin{Verbatim}[samepage=true,xleftmargin=15mm,baselinestretch=0.8]
 java mouse.Generate -G myGrammar.txt -P myParser
\end{Verbatim}
\normalsize
%      
This should produce the answer: 
%      
\small
\begin{Verbatim}[samepage=true,xleftmargin=15mm,baselinestretch=0.8]
 2 rules
 1 unnamed
 3 terminals
\end{Verbatim}
\normalsize
%      
and generate in the \tx{work} directory a file \tx{myParser.java}
containing class \tx{myParser}.
This is your parser.

If you open the file in a text editor, you will see that the 
parsing expressions \tx{Sum} and \tx{Number} have been transcribed
into parsing procedures in the form of methods in the class \tx{myParser}. 
The unnamed inner expression \tx{("+"~Number)} has been transcribed
into the method \verb#Sum_0#:

\small
\begin{Verbatim}[frame=single,framesep=2mm,samepage=true,xleftmargin=15mm,xrightmargin=15mm,baselinestretch=0.8]
  //=====================================================================
  //  Sum = Number ("+" Number)* !_ ;
  //=====================================================================
  private boolean Sum()
    {
      begin("Sum");
      if (!Number()) return reject();
      while (Sum_0());
      if (!aheadNot()) return reject();
      return accept();
    }
  
  //-------------------------------------------------------------------
  //  Sum_0 = "+" Number
  //-------------------------------------------------------------------
  private boolean Sum_0()
    {
      begin("");
      if (!next('+')) return rejectInner();
      if (!Number()) return rejectInner();
      return acceptInner();
    }
  
  //=====================================================================
  //  Number = [0-9]+ ;
  //=====================================================================
  private boolean Number()
    {
      begin("Number");
      if (!nextIn('0','9')) return reject();
      while (nextIn('0','9'));
      return accept();
    }
\end{Verbatim}
\normalsize

\medskip
The methods \tx{begin()}, \tx{accept()}, \tx{reject()}, \tx{next()}, etc.
are standard services provided by \Mouse.
The first three maintain, behind the scenes, the information
needed to define semantics;
it will be discussed in the next section.
As you may guess, \tx{accept()} returns \emph{true} and \tx{reject()} returns \emph{false}.
The methods \tx{next()} and \tx{nextIn()} implement terminals.
For example, \tx{next('+')} implements the parsing expression \tx{"+"}:
if the next input character is \tx{+}, consumes it and returns \emph{true},
otherwise returns \emph{false}.

\newpage
As you can see by inspecting the file, class \tx{myParser} is defined as a subclass of
\tx{mouse.runtime.ParserBase} provided by \Mouse.
The service methods are inherited from that superclass.
The structure of your parser is thus as follows:

%----------------------------------------------------------------------
\small
\begin{center}
\begin{texdraw}
  \drawdim {mm}
  \setunitscale 0.75
  \linewd 0.2
  \textref h:L v:B
  \arrowheadtype t:F
  \arrowheadsize l:2 w:1
  %
  \move(0 0)
  \savecurrpos(*mPx *mPy)
  \rlvec(40 0) \rlvec(0 40) \rlvec(-40 0) \rlvec(0 -40)
  \move(*mPx *mPy) \rmove(2 6)  \htext{\tx{myParser}}
                   \rmove(0 -4) \htext{\tx{(generated)}}
  \move(*mPx *mPy) \rmove(20 40) \rlvec(0 10) \rmove(0 -3) \rlvec(-3 -3) \rlvec(6 0) \rlvec(-3 3)
  \move(*mPx *mPy) \rmove(6 20)   \htext{\tx{parsing}} 
                   \rmove(0 -3.5) \htext{\tx{procedure}}
  \move(*mPx *mPy) \rmove(12 26) \lpatt(2 1) \ravec(0 28) \lpatt()                 
  %
  \move(0 50)
  \savecurrpos(*bPx *bPy)
  \rlvec(40 0) \rlvec(0 40) \rlvec(-40 0) \rlvec(0 -40)
  \move(*bPx *bPy) \rmove(2 35)   \htext{\tx{ParserBase}}
                   \rmove(0 -4)   \htext{\tx{(Mouse)}}
  \move(*bPx *bPy) \rmove(6 15)   \htext{\tx{begin}} 
                   \rmove(0 -2.5) \htext{\tx{next}}
                   \rmove(0 -3.5) \htext{\tx{accept}}
                   \rmove(0 -2.5) \htext{\tx{etc.}}
\end{texdraw}
\end{center}
\normalsize
%----------------------------------------------------------------------

Compile your parser by typing:      
       
\small
\begin{Verbatim}[samepage=true,xleftmargin=15mm,baselinestretch=0.8]
 javac myParser.java
\end{Verbatim}
\normalsize
       
To try the parser, type:
       
\small
\begin{Verbatim}[samepage=true,xleftmargin=15mm,baselinestretch=0.8]
 java mouse.TryParser -P myParser
\end{Verbatim}
\normalsize
  
This invokes a \Mouse\ tool for running the parser.
It responds with "\tx{> }" and waits for input.
If you enter a correct line, 
it prints another "\tx{> }" to give you another chance.
If your entry is incorrect, you get an error message
before another "\tx{> }".
You exit by pressing Enter just after "\tx{> }".

The session may look like this:
 
\small
\begin{Verbatim}[samepage=true,xleftmargin=15mm,baselinestretch=0.8]
 java mouse.TryParser -P myParser
 > 1+2+3
 > 17+4711
 > 2#2
 After '2': expected [0-9] or '+' or end of text
 >
\end{Verbatim}
\normalsize
