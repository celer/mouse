%HHHHHHHHHHHHHHHHHHHHHHHHHHHHHHHHHHHHHHHHHHHHHHHHHHHHHHHHHHHHHHHHHHHHHHHHHH

\section{Parsing Expression Grammar\label{PEG}}

%HHHHHHHHHHHHHHHHHHHHHHHHHHHHHHHHHHHHHHHHHHHHHHHHHHHHHHHHHHHHHHHHHHHHHHHHHH

\subsection{Parsing expressions}

Parsing expressions are instructions for parsing strings,
written in a special language.
You can think of a parsing expression as shorthand for a procedure 
that carries out such instruction.
The expressions can invoke each other recursively, thus forming
together a recursive-descent parser.

In general, parsing expression is applied to a character string
(the "input" to be parsed)
at a position indicated by some "cursor".
It tries to recognize a portion of the string ahead of the cursor.
Usually, it "consumes" the recognized portion by advancing the cursor
and indicates "success"; 
if it fails at the recognition, it indicates "failure" 
and does not consume anything (does not advance the cursor).

The following five kinds of parsing expressions work directly
on the input string:

\ul
\item[$\tx{"}s\tx{"}$]\upsp\newline 
     where $s$ is a nonempty character string.
     If the text ahead starts with the string $s$,
     consume that string and indicate success.
     Otherwise indicate failure.\dnsp

\item[\tx{[}$s$\tx{]}]\upsp\newline 
     where $s$ is a nonempty character string.
     If the text ahead starts with a character appearing in $s$,
     consume that character and indicate success.
     Otherwise indicate failure.\dnsp

\item[\tx{\textasciicircum[}$s$\tx{]}]\upsp\newline 
     where $s$ is a nonempty character string.
     If the text ahead starts with a character \emph{not} appearing in $s$,
     consume that character and indicate success.
     Otherwise indicate failure.\dnsp

\item[$\tx{[}c_1\tx{-}\,c_2\tx{]}$]\upsp\newline 
     where $c_1,c_2$ are two characters.
     If the text ahead starts with a character from the range 
     $c_1$~through~$c_2$, consume that character and indicate success.
     Otherwise indicate failure.\dnsp

\item[\textbf{\_\ \ }]\upsp\newline
     where \textbf{\_\;} is the underscore character.
     If there is a character ahead, consume it
     and indicate success.
     Otherwise (that is, at the end of input)
     indicate failure.\dnsp
\eul

These expressions are analogous to terminals of a classical context-free
grammar, and are in the following referred to as "terminals".
The remaining kinds of parsing expressions
invoke other expressions to do their job:

\ul

\item[$e\tx{?}$]\upsp\newline
     Invoke the expression $e$ and indicate success whether it succeeded or not.\dnsp

\item[$e\tx{*}$]\upsp\newline
     Invoke the expression $e$ repeatedly as long as it succeeds.\newline
     Indicate success even if it did not succeed a single time.\dnsp

\item[$e\tx{+}$]\upsp\newline
     Invoke the expression $e$ repeatedly as long as it succeeds.\newline
     Indicate success if $e$ succeeded at least once.
     Otherwise indicate failure.\dnsp

\item[$\tx{\&}e$]\upsp\newline
     This is a lookahead operation ("syntactic predicate"):
     invoke the expression $e$ and then
     reset the cursor as it was before the invocation of $e$
     (do not consume input).\newline
     Indicate success if $e$ succeeded or failure if $e$ failed.\dnsp

\item[$\tx{!}e$]\upsp\newline
     This is a lookahead operation ("syntactic predicate"):
     invoke the expression $e$ and then
     reset the cursor as it was before the invocation of $e$
     (do not consume input).\newline
     Indicate success if $e$ failed or failure if $e$ succeeded.\dnsp
     
\newpage

\item[$e_1 \ldots e_n$]\upsp\newline
     Invoke expressions $e_1,\ldots,e_n$, in this order,
     as long as they succeed.
     Indicate success if all succeeded.\newline 
     Otherwise reset cursor as it was before the invocation of $e_1$
     (do not consume input) and indicate failure.\dnsp

\item[$e_1\,\tx{/}\ldots\,\tx{/}\,e_n$]\upsp\newline
     Invoke expressions $e_1,\ldots,e_n$, in this order,
     until one of them succeeds. 
     Indicate success if one of expressions succeeded.
     Otherwise indicate failure.\dnsp

\eul

The following two expressions are shorthand forms of frequently used constructions.
In addition to being easier to read, they have a more efficient implementaton.

\ul

\item[$e_1\tx{*+}\,e_2$]\upsp\newline
     A shorthand for \tx{(!}$e_2\,e_1$\tx{)*}\,$e_2$:
     iterate $e_1$ until $e_2$.
     More precisely: invoke $e_2$; if it fails, invoke $e_1$ and try $e_2$ again.
     Repeat this until $e_2$ succeeds.
     If any of the invocations of $e_1$ fails,
     reset cursor as it was at the start
     (do not consume input) and indicate failure.\dnsp

\item[$e_1\tx{++}\,e_2$]\upsp\newline
     A shorthand for \tx{(!}$e_2\,e_1$\tx{)+}\,$e_2$, 
     or \tx{(!}$e_2\,e_1$\tx{)(!}$e_2\,e_1${\tx)*}\,$e_2$:
     iterate $e_1$ at least once until $e_2$.
     Invoke $e_2$. If it succeeds, reset the cursor (do not consume input) and indicate failure.
     Otherwise invoke $e_1$ and try $e_2$ again.
     Repeat this until $e_2$ succeeds.
     If any of the invocations of $e_1$ fails,
     reset cursor as it was at the start
     (do not consume input) and indicate failure.\dnsp

\eul

\medskip
The following table summarizes all forms of parsing expressions.

% ---------------------------------------------------------------------
\begin{center}
\begin{tabular}{|c|l|c|} \hline
expression & name & precedence\upsp\dnsp \\ \hline
$\tx{"}s\tx{"}$  & String Literal \upsp & 5 \\
$\tx{[}s\tx{]}$  & Character Class & 5 \\
$\tx{\textasciicircum[}s\tx{]}$  & Not Character Class & 5 \\
$\tx{[}c_1\tx{-}�\,c_2\tx{]}$ & Character Range & 5 \\
\textbf{\_} & Any Character & 5 \\
$e\tx{?}$  & Optional & 4 \\
$e\tx{*}$  & Iterate & 4 \\
$e\tx{+}$  & One or More & 4 \\
$e_1\tx{*+}e_2$  & Iterate Until & 4 \\
$e_1\tx{++}e_2$  & One or More Until & 4 \\
$\tx{\&}e$ & And-Predicate & 3 \\
$\tx{!}e$  & Not-Predicate & 3 \\
$e_1 \ldots e_n$  & Sequence & 2 \\
$e_1\,\tx{/}\ldots\,\tx{/}\,e_n$  & Choice\dnsp & 1 \\ \hline
\end{tabular}
\end{center}
% ---------------------------------------------------------------------


The expressions $e, e_1, \ldots, e_n$ above can be specified either explicitly
or by name (the way of naming expressions will be explained in a short while).
An expression specified explicitly in another expression
with the same or higher precedence
must be enclosed in parentheses.

Backtracking takes place in the Sequence expression.
If $e_1,\ldots,e_i$ in $e_1\ldots e_i\ldots e_n$ succeed and consume some input,
and then $e_{i\,+1}$ fails, the cursor is moved back to where it was before trying $e_1$;
the whole expression fails.
If this expression was invoked from a Choice expression (and was not the last there),
Choice has an opportunity to try another alternative on the same input.

However, the opportunities to try another alternative are limited:
once $e_i$ in the Choice expression \tx{$e_1/\ldots/e_i/\ldots /e_n$} succeeded, 
none of the alternatives $e_{i\,+1},\ldots e_n$ will ever be tried on the same input,
even if the parse fails later on.

\subsection{The grammar}

Parsing Expression Grammar is a list of one or more "rules" of the form:
%
\begin{equation*}
\textit{name}\quad \tx{=}\quad \textit{expr} \;\;\tx{;}
\end{equation*}
%
where \textit{expr} is a parsing expression,
and \textit{name} is a name given to it.
The \textit{name} is a string of one or more letters (\tx{a-z}, \tx{A-Z}) and/or digits,
starting with a letter. 
White space is allowed everywhere except inside names.
Comments starting with a double slash and extending to the end of a line are also allowed.

The order of the rules does not matter, except that the expression specified first
is the "top expression", invoked at the start of the parser.

\medskip
A specific grammar may look like this:

\smallskip
\small
\begin{Verbatim}[frame=single,framesep=2mm,samepage=true,xleftmargin=15mm,xrightmargin=15mm,baselinestretch=0.8]
   Sum    = Number ("+" Number)* !_ ;
   Number = [0-9]+ ;
\end{Verbatim}
\normalsize
%
It consists of two named expressions: \tx{Sum} and \tx{Number}.
They define a parser consisting of
two procedures named \tx{Sum} and \tx{Number}.
The parser starts by invoking \tx{Sum}.
The \tx{Sum} invokes \tx{Number}, and if this succeeds,
repeatedly invokes \tx{("+" Number)} as long as it succeeds.
Finally, \tx{Sum}
invokes a sub-procedure for the predicate "\verb#!_#",
which succeeds only if it does not see any character ahead --
that is, only at the end of input.
The \tx{Number} reads digits in the range from 
0 through 9 as long as it succeeds,
and is expected to find at least one such digit.

One can easily see that the parser
accepts strings like "\tx{2+2}",  "\tx{17+4711}", or "\tx{2}";
in general, one or more integers separated by "\tx{+}".
You could guess this by reading the above as EBNF,
which often leads you right.
However, you should keep in mind that,
in spite of its outward similarity,
PEG \emph{is not} EBNF.

If you are not familiar with PEG programming, you may get some insights
into its problems and pitfalls from the author's papers 
\cite{Redz:2007:FI,Redz:2008:FI,Redz:2009:FI}.

It is quite obvious that the grammar you define must not be left-recursive,
and the expression $e$ in $e\tx{*}$ and $e\tx{+}$ must never succeed
in consuming an empty string;
otherwise, the parser could run into an infinite descent or an infinite loop.
In \cite{Ford:2004} Ford defined a formal property \textit{WF} (Well-Formed)
that guarantees absence of these problems.
\Mouse\ computes this property and points out expressions that are not well-formed.

