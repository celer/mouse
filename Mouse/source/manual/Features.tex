%HHHHHHHHHHHHHHHHHHHHHHHHHHHHHHHHHHHHHHHHHHHHHHHHHHHHHHHHHHHHHHHHHHHHHHHHHH

\section{Miscellaneous features}

%HHHHHHHHHHHHHHHHHHHHHHHHHHHHHHHHHHHHHHHHHHHHHHHHHHHHHHHHHHHHHHHHHHHHHHHHHH

This section describes some features not mentioned in the tutorial.

\subsection{Escaped characters\label{escape}}

None of the characters you write in String Literal, Character Class, and Character Range
is allowed to be newline or carriage return.
The same applies to diagnostic names in pointed brackets.
It is also not advisable to use there any white-space characters other than blank.
Other characters, namely \tx{"}, \tx{]}, \tx{-}, \tx{>}, cannot be used because they will be mistaken
for delimiters. 
Still other characters may be not available at your keyboard.
You specify all such characters using the conventions of "escape". 

Any character other than \tx{n}, \tx{r}, \tx{t}, or \tx{u},
preceded by a backslash \verb#\# represents itself as part of the string $s$
in String Literal and Character Class,
or as $c_1, c_2$ in Character Range, but is not recognized as a delimiter.
Thus, for example, \verb#"\""# is a String Literal of one double quote~\tx{"},
and \verb#[[\]]# is a Character Class consisting of the brackets \tx{[} and~\tx{]}.

The special combinations \verb#\n#, \verb#\r#, and \verb#\t# represent
the newline, carriage return, and tab characters, respectively.
The special combination \verb#\u# must be followed by four hexadecimal digits
representing a Unicode character. 
Thus, for example, \verb#\u000B# stands for the vertical tab character.


\subsection{Character encoding}

Both your grammar file and the file processed by your parser 
consist of bytes
that may be interpreted in different ways
depending on the used \emph{character encoding scheme}.
For example, the byte with value 248 (hex \tx{f8}) means the Norwegian character \tx{\o} 
in the encoding ISO-8859-1, or the Czech character \tx{\v{r}}
in the encoding ISO-8859-2;
the command prompt window of Windows XP displays it as a degree sign~$ ^\circ$.
The bytes with values in the range 32 through 126, known as the ASCII character set,
have a fixed interpretation that does not depend of encoding.

Both the \tx{Generate} tool and the resulting parser use the default 
character encoding set up for the Java Virtual Machine executing them.
%
Your grammar uses, in principle, only the ASCII characters, 
but you can have other characters in values of terminals, 
in diagnostic names, or in comments.
These chracters are interpreted according to the default encoding.
When they appear as literals in the generated Java file,
they are converted to Unicode escapes that are composed of ASCII characters.
It means that the compiled parser does not depend on the 
encoding used at the compile time.

The input file processed by the parser is also interpreted 
according to the default encoding.
If you need, you may enforce a specific encoding for input files
by modifying the wrapper \tx{SourceFile}.


\subsection{Reserved names}

The names you give to expressions and syntactic actions are used as  
method and object names in the generated parser.
Therefore they must be distinct from Java reserved words.
They may also conflict with names used in the base classes,
but no checks are made in the present version, and
\textbf{you have, unfortunately, to watch yourself for possible conflicts!}
 

\subsection{Boolean actions}

You may specify semantic actions that return a \tx{boolean} value
in addition to inserting semantic value in the left-hand side \Phrase.
The action returning \tx{false} causes the invoking procedure to fail.
You specify such action by placing an ampersand in front of the action name.
For example, the action \tx{identifier()} invoked by \tx{Identifier} below
may return \tx{false} if the recognized text is one of reserved words,
thus forcing \tx{Identifier} to fail on reserved words:

\small
\begin{Verbatim}[frame=single,framesep=2mm,samepage=true,xleftmargin=15mm,xrightmargin=15mm,baselinestretch=0.8]
   Identifier = Letter (Letter / Digit)*  {& identifier} ;
\end{Verbatim}
\normalsize


\subsection{Omitted action names}

To avoid clobbering your grammar with names of semantic actions, 
you may only indicate the presence of an action with an empty pair of brackets \verb#{}#.
The semantic action thus specified has the same name as the rule.
The actions for multiple alternatives of a choice expression will have numbers attached to them.
For example, you can write:

\small
\begin{Verbatim}[frame=single,framesep=2mm,samepage=true,xleftmargin=15mm,xrightmargin=15mm,baselinestretch=0.8]
   Product = Factor (MultOp Factor)*  {} ;
   Factor  = Digits? "." Digits Space {}
           / Digits Space             {}
           / Lparen Sum Rparen        {} 
           / Name Space               {} ; 
\end{Verbatim}
\normalsize

The semantic actions defined here are
\tx{Product()} and \verb#Factor_0()# through \verb#Factor_3()#.
These names will appear in the skeleton generated by \tx{-s} option.

You specify a Boolean action by an ampersand in brackets: \verb#{&}#.

You specify an error action by brackets preceded with a tilde: \verb#~{}#.\newline
The name of the action thus specified has suffix \verb#"_fail"#.


\subsection{Own diagnostic output}

The error messages described in Section~\ref{errors}
are written to \tx{System.out}.
If this does not suit you, you may yourself take care of the message
as shown in Section~\ref{errRec}.
You define a semantic action to be called on failure of the top procedure.
In that action, you obtain the error message using \tx{lhs().errMsg()},
and handle it in the way you need.
You should then issue \tx{lhs().errClear()} to prevent the message from
being written to \tx{System.out}.

The messages returned by \tx{errMsg()} may contain unprintable
characters appearing in your String Literals, Character Classes, and Character Ranges.
In the \tx{String} returned by \tx{errMsg()},
characters outside the range \verb#\u0020# through \verb#\u00ff#
are replaced by Java-like
escapes such as \verb#\n# for a newline character, 
or \verb#\u03A9# for the Greek letter $\Omega$.
Characters in the range \verb#\u007f# through \verb#\u00ff# are not replaced;
they may be displayed differently depending on the encoding used to view the message.

(The above applies to all messages from the parser.)


\subsection{Trace}

The base class \tx{mouse.runtime.SemanticsBase} contains a \tx{String} variable
\tx{trace}, normally initialized to empty string.
Using the method \tx{setTrace()} of your parser, 
or the option \tx{-T} on \Mouse\ tools \tx{TryParser} and \tx{TestParser},
you can assign a value to this variable in order to trigger special actions
such as tracing.

Note that the same value is used to trigger traces in the test version
of the parser; therefore, you should not use strings containing 
letters \tx{e}, \tx{i}, or \tx{r} if you are going to use this feature
together with the test version.


\subsection{Package name}

In a real application, your parser and semantics classes will belong to some
named package.
You specify the name of that package using option \tx{-p} on 
\tx{mouse.Generate}.

\subsection{Initialization}

The base class \tx{mouse.runtime.SemanticsBase} contains an empty method
\tx{void Init()} that is invoked every time you start your parser
by invoking its \tx{parse} method.
By overriding this method in your semantics class,
you can provide your own initialization of that class.


\subsection{Specifying output directory}

By default, the \tx{Generate} tool places the resulting parser
(and semantics skeleton if requested) in the current work directory.
This default has been used in all the examples.
You can specify any output directory of your choice using the \tx{-D}
option on \tx{mouse.Generate}.
The name of this directory need not correspond to the specified package name.

\subsection{Checking the grammar}

As mentioned in Section~\ref{PEG}, \Mouse\ checks if the grammar is well-formed,
thus ensuring that the generated parser will always stop.
As an example, the following grammar is not well-formed:

\small
\begin{Verbatim}[frame=single,framesep=2mm,samepage=true,xleftmargin=15mm,xrightmargin=15mm,baselinestretch=0.8]
   A = (!"a")* / "b"? A ;
\end{Verbatim}
\normalsize

As \tx{"b"?} may succeed without consuming any input, there is a possibility 
of infinite descent for \tx{A}. 
As \tx{!"a"} may succeed without consuming input, there is a possibility
infinite iteration in \tx{(!"a")*}.
The \tx{Generate} tool produces these messages:

\small
\begin{Verbatim}[samepage=true,xleftmargin=15mm,baselinestretch=0.79]
   Warning: the grammar not well-formed.
   - !"a" in (!"a")* may consume empty string.
   - A is left-recursive via "b"? A.
\end{Verbatim}
\normalsize

You may use the \Mouse\ tool \tx{TestPEG} to check the grammar without generating parser.
By specifying option \tx{-D} to the tool, you display the grammar
together with attributes that were used to compute well-formedness according to \cite{Ford:2004}.
(The rules for computing these attributes can also be found 
as Figs. 5 and 6 in \cite{Koprowski:Binsztok:2010}.)

\small
\begin{Verbatim}[samepage=true,xleftmargin=15mm,baselinestretch=0.79]
   java mouse.TestPEG -G badGrammar.txt -D    
   Warning: the grammar not well-formed.
   - !"a" in (!"a")* may consume empty string.
   - A is left-recursive via "b"? A.

   1 rules
     A = (!"a")* / "b"? A ;   //  0 !WF

   4 subexpressions
     (!"a")*   //  0 !WF
     "b"?   //  01
     !"a"   //  0f
     "b"? A   //  01 !WF

   2 terminals
     "a"   //  1f
     "b"   //  1f
\end{Verbatim}
\normalsize

The symbols appearing as comments on the right have these meanings:
\ul
\item[] \tx{0} = may succeed without consuming input;
\item[] \tx{1} = may succeed after consuming some input;
\item[] \tx{f} = may fail;
\item[] \tx{!WF} = not well-formed.
\eul


